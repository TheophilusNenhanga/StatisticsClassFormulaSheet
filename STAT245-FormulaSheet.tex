\documentclass{article}
\usepackage[a4paper, margin=2cm, marginparwidth=1cm]{geometry}
\usepackage{amsmath}
\usepackage[colorlinks=true, allcolors=blue]{hyperref}

\title{STAT 245 FORMULA SHEET}
\author{Theophilus Nyasha Nenhanga}
\date{June 2023}

\begin{document}

\maketitle

\section{Introduction}

\subsection{Mean}

\begin{align*}
    \text{Mean} &= \frac{\text{Sum of observations}}{\text{Number of Observations}}\\[12pt]
    \mu &= \text{Population Mean}\\
    \overline{x} &= \text{Sample Mean}
\end{align*}

\begin{align*}
    \text{Population Mean, } \mu &= \frac{\sum x_{i}}{N}\\
    \text{Sample Mean, } \overline{x} &= \frac{\sum x_{i}}{n}
\end{align*}

\subsection{Median}
\begin{align*}
    \text{Median, } &M\\
    \text{Position of Median} &= \frac{n+1}{2}
\end{align*}

\subsection{Measures of Spread}

\subsubsection{Range}

\begin{align*}
    \text{Range, } &R\\
    R = \text{largest value} &- \text{Smallest value}
\end{align*}

\subsubsection{Mean Deviation}

\begin{align*}
    \text{Mean Deviation, } MD\\
    \text{For Populations, } \frac{1}{N} \sum_{i = 1}^{N} \left\lvert x_{i} - \mu\right\rvert\\[12pt]
    \text{For Samples, } \frac{1}{n} \sum_{i = 1}^{N} \left\lvert x_{i} - \overline{x}\right\rvert
\end{align*}

\begin{align*}
    \text{N} &= \text{Population size}\\
    \mu &= \text{Population mean}\\
    n &= \text{Sample size}\\
    \overline{x} &= \text{sample mean}
\end{align*}

\subsubsection{Variance}

\begin{align*}
    \text{For Population Variance} \sigma^2\\
    \sigma^2 = \frac{1}{N} \left( \sum x_{i}^2 - \frac{1}{N} {\left(\sum x_{i}\right)}^2 \right)\\[12pt]
    \text{For Sample Variance} s^2\\
    s^2 = \frac{1}{N} \left( \sum x_{i}^2 - \frac{1}{n-1} {\left(\sum x_{i}\right)}^2 \right)
\end{align*}

\subsubsection{Standard Deviation}

\begin{align*}
    \text{For Population Variance} \sigma\\
    \sigma = \sqrt{\sigma^2}\\
    \text{For Sample Variance} s\\
    s = \sqrt{s^2}
\end{align*}

\subsection{Dealing with large Samples}
\begin{align*}
    \text{Population Size, } N &= \sum f_{i} \\[12pt]
    \text{Mean, } \mu &= \frac{\sum x}{N} = \frac{\sum (x_{i}f_{i})}{\sum f_{i}} = \frac{\sum (x_{i}f_{i})}{N} \\[12pt]
    \text{Variance, } \sigma^2 &= \frac{\sum x^2 - \frac{{(\sum x)}^2}{N}}{N} = \frac{\sum ({x_{i}}^2f_{i}) - \frac{{\sum (x_{i}f_{i})}^2}{\sum f_{i}}}{\sum f_{i}} = \frac{\sum ({x_{i}}^2f_{i}) - \frac{{\sum (x_{i}f_{i})}^2}{N}}{N}
\end{align*}

\section{Bell Shaped Distributions}

\subsection{The z-score}

\begin{align*}
    &\text{The z score is a multiple of how many standard deviatiosn we are away from the mean.}\\
    & z = \frac{x - \mu}{\sigma}
\end{align*}

\section{Margin of Error}

\begin{align*}
    ME &= z_{\frac{\alpha}{2}}\sqrt{\frac{p \cdot q}{n}}\\
    p &= \text{proportion} \\
    n &= \text{sample size} \\
    z &= \text{critical value}\\ 
    \alpha &= \text{confidence interval}     
\end{align*}

\section{Confidence Intervals}

\subsection{When Mean Unknown, Standard Deviation and Sample Size Given}

\begin{align*}
    2 \cdot z_{\frac{\alpha}{2}} \cdot \frac{\sigma}{\sqrt{n}}
\end{align*}

\subsection{Sample Mean and Sample Standard Deviation Given}

\begin{align*}
    \overline{x} -  z_{\frac{\alpha}{2}} \cdot \frac{\sigma}{\sqrt{n}} &\leq \mu \leq \overline{x} +  z_{\frac{\alpha}{2}} \cdot \frac{\sigma}{\sqrt{n}}\\
    \overline{x} &= \text{sample mean}\\
    n &= \text{sample size}\\
    s &= \text{sample standard deviation}\\
    \alpha &= \text{confidence interval}
\end{align*}

\subsection{Binomial Populations}

\begin{align*}
    \hat{p} -  z_{\frac{\alpha}{2}} \cdot \sqrt{\frac{\hat{p}\hat{q}}{n}} &\leq \mu \leq \hat{p} +  z_{\frac{\alpha}{2}} \cdot \sqrt{\frac{\hat{p}\hat{q}}{n}}\\
    \hat{p} &= \text{proportion}\\
    \hat{q} &= 1 - \hat{p}\\
    n &= \text{sample size}\\
    \alpha &= \text{confidence interval}
\end{align*}

\section{Two Sample Problems}
\subsection{Difference Between Population Means}
\begin{align*}
    \overline{x}_{1} - \overline{x}_{2} - z_{\frac{\alpha}{2}} \cdot \sqrt{\frac{\sigma^2_{1}}{n_{1}}+\frac{\sigma^2_{2}}{n_{2}}}  &\leq  \mu_{1} - \mu_{2} \leq \overline{x}_{1} - \overline{x}_{2} + z_{\frac{\alpha}{2}} \cdot \sqrt{\frac{\sigma^2_{1}}{n_{1}}+\frac{\sigma^2_{2}}{n_{2}}} \\ 
    \overline{x}_{1} &= \text{sample mean}_{1}\\
    \overline{x}_{2} &= \text{sample mean}_{2}\\
    \sigma^2_{1} &= \text{variance}_{1}\\
    \sigma^2_{2} &= \text{variance}_{2}\\
    \alpha &= \text{confidence interval}\\ 
    n_{1} &= \text{sample size}_{1}\\
    n_{2} &= \text{sample size}_{2}
\end{align*}


\subsection{Difference Between Population Proportions}
\begin{align*}
    \hat{p}_{1} - \hat{p}_2 - z_{\frac{\alpha}{2}} \cdot \sqrt{ \frac{\hat{p}_{1} \cdot \hat{q}_{1}}{n_{1}} + \frac{\hat{p}_{2} \cdot \hat{q}_{2}}{n_{2}} }  &\leq  \hat{p}_{1} - \hat{p}_{2} \leq \hat{p}_{1} - \hat{p}_2 + z_{\frac{\alpha}{2}} \cdot \sqrt{ \frac{\hat{p}_{1} \cdot \hat{q}_{1}}{n_{1}} + \frac{\hat{p}_{2} \cdot \hat{q}_{2}}{n_{2}} } \\ 
    \hat{p}_{2} &= \text{sample proportion}_{1}\\
    \hat{p}_{1} &= \text{sample proportion}_{2}\\
    \hat{q}_{1} &= 1 - \hat{p}_{1}\\ 
    \hat{q}_{2} &= 1 - \hat{p}_{2}\\ 
    \alpha &= \text{confidence interval}\\ 
    n_{1} &= \text{sample size}_{1}\\
    n_{2} &= \text{sample size}_{2}
\end{align*}

\section{Estimating The Population Mean}

\subsection{Estimating The Population Mean}

\begin{align*}
    n \geq \left( \frac{}{} \right)
\end{align*}

\section{Pooled Variables}

\subsection{Pooled Variance}

\begin{align*}
    \frac{(n_{1}-1) \cdot s^2_{1} + (n_{2}-1) \cdot s^2_{2}}{n_{1} + n_{2} - 2}
\end{align*}

\section{Hypothesis Testing}
\subsection{When to reject}
\subsubsection{Critical Value Approach}
\begin{align*}
    \text { 1. If it is a two tailed test, Critical Values are } -z_{\frac{\alpha}{2}} \text{ and } z_{\frac{\alpha}{2}} \\
    \bullet \text { Reject } H_{0} \text{ if TS } \leq -z_{\frac{\alpha}{2}} \text{or TS} \geq z_{\frac{\alpha}{2}} \\
    \bullet \text { Do not reject } H_{0} \text{ if } -z_{\frac{\alpha}{2}} < \text{TS} < z_{\frac{\alpha}{2}} 
\end{align*}

\begin{align*}
    \text { 2. If it is a left tailed test, Critical Value is } -z_{\alpha} \\
    \bullet \text { Reject } H_{0} \text{ if TS } \leq -z_{\alpha} \\
    \bullet \text { Do not reject } H_{0} \text{ if }  \text{ TS } > -z_{\frac{\alpha}{2}}
\end{align*}

\begin{align*}
    \text { 3. If it is a right tailed test, Critical Value is } z_{\alpha} \\
    \bullet \text { Reject } H_{0} \text{ if TS } \geq -z_{\alpha} \\
    \bullet \text { Do not reject } H_{0} \text{ if }  \text{ TS } < z_{\frac{\alpha}{2}}
\end{align*}

\subsubsection{P Value Approach}
\begin{itemize}
    \item if it is a two-tailed test, then \[ P Value = P(z \leq - |TS|) + P(z \geq |TS|) = 2 \cdot P(z \leq -|TS|) \]
    \item If it is a left-tailed test, then \[ P Value = P(z \leq TS) \]  
    \item If it is a left-tailed test, then \[ P Value = P(z \leq TS) \]
\end{itemize}
And For all three types of test, 
\begin{itemize}
    \item Reject $ H_{0} $ if $ PValue \leq \alpha $
    \item Do not reject $ H_{0} $ if $ PValue > \alpha $
\end{itemize}

\section{Calculating Test Statistics}

\subsection{Testing One Mean}

\begin{align*}
    \text{TS} &= \frac{x-\mu_{0}}{\frac{s}{\sqrt{n}}} \\ 
    \text{TS} &= \text{Test Statistic}
\end{align*}

\subsection{Testing One Proportion}
\begin{align*}
    \text{TS} = \frac{\hat{p} - p_{0}}{\sqrt{\frac{p_{0}\cdot q_{0}}{{n}}}}
\end{align*}


\end{document}
